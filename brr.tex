\documentclass{beamer}
\mode<presentation>
{
  \usetheme{Madrid}      % or try Darmstadt, Madrid, Warsaw, ...
  \usecolortheme{crane} % or try albatross, beaver, crane, ...
  \usefonttheme{default}  % or try serif, structurebold, ...
  \setbeamertemplate{navigation symbols}{}
  \setbeamertemplate{caption}[numbered]
} 

\usepackage[english,russian]{babel}
\usepackage[T2A]{fontenc}
\usepackage[utf8]{inputenc}
\usepackage{tikz,pgfopts}
\usetikzlibrary{arrows,shapes}
\usepackage{pgfplots}
\usetikzlibrary{positioning}
\usepackage{amsmath,amscd}
\usepackage{wrapfig}
\usepackage{verbatim}
\usepackage[figurename=Расм ]{caption}
%\usepackage[utf8x]{inputenc}



\author{Ashurov Sindor}
\title{\texttt{Вывод из эксплуатации объектов ядерной энергетики в Великобритании
}}
\date{\today}

\begin{document}

\begin{frame}
  \titlepage
\end{frame}

\begin{frame}{Что такое ядерные отходы?}

В процессе выработки электроэнергии задействованы не только атомные  электростанции, но и многие другие сооружения и установки. В результате их деятельности образуется большой объем ядерных отходов. Ядерные отходы – радиоактивны и опасны. Они содержат атомные ядра, которые со временем произвольно трансформируются (вследствие реакции распада или расщепления) в иные атомные ядра. В ходе таких трансформаций радиоактивные отходы выделяют ионизирующее излучение. Это так называемое $\alpha$-, $\beta$-, $\gamma$– или нейтронное излучение, которое имеет способность вторгаться в любую материю на своём пути, ионизировать и тем самым разрушать её.

\end{frame}


\begin{frame}{Как образуются ядерные отходы?}
    Можно выделить пять главных источников образования ядерных отходов при использовании ядерной энергии:
    \begin{enumerate}
        \item  ядерные отходы, образующиеся при добыче и обработке урановой руды,
\item ядерные отходы, образующиеся при эксплуатации АЭС
\item ядерные отходы, образующиеся во время переработки отработанного топлива
\item  ядерные отходы, образующиеся при выводе из эксплуатации ядерных объектов
\item ядерные отходы, образующиеся при обращении с радиоактивными отходами.
   \end{enumerate}
\end{frame}

\begin{frame}{Классификация радиоактивных отходов}

Классификация радиоактивных отходов у стран участниц ЕС различна. Примерами критерий для классификации могут быть: насколько долго отходы будут оставаться на опасном уровне радиоактивности, какова концентрация радиоактивного материала в отходах, и вырабатывают ли они тепло. Комиссия Европейского Союза разработала рекомендации для системы классификации твердых радиоактивных отходов. Однако страны-участницы еще не внедрили ее полностью. Основными категориями радиоактивных отходов являются:
\begin{itemize}
\item Отходы очень низкого уровня (ОНУО)
\item Отходы низкого и среднего уровня (НСУО или НУО и СУО)
\item Отходы высокого уровня (ВУО)
\end{itemize}

\end{frame}

\begin{frame}{История развития ядерного комплекса в Великобритании
}

Решение о запуске собственной программы ядерных вооружений было принято правительством Великобритании после окончания Второй мировой войны. Для этих целей был создан комплекс Селлафилд, миссия которого состояла в производстве и переработке оружейного плутония. 
Работы по сооружению первых двух реакторов с газовым охлаждением Уиндскейл Пайл 1 и Уиндскейл Пайл 2 стартовали в 1947 году. Первые два реактора «Уиндскейл Пайл», представлявшие собой разновидность графитового реактора с воздушным охлаждением, использовались исключительно для наработки оружейного плутония.
С 1951 по 1957 год завод В204 произвел 385 кг плутония, примененного при создании первой британской атомной бомбы, испытанной на островах Монте-Белло в Австралии в октябре 1952 года. 
\end{frame}

\begin{frame}{История развития ядерного комплекса в Великобритании}
    Так, Великобритания стала третьей после США и России атомной сверхдержавой.
До 1972 года в Великобритании фактически не существовало разделения между гражданскими и военными ядерными программами. Первые оксид-магниевые реакторы «Магнокс» одновременно производили электроэнергию и нарабатывали оружейный плутоний. 
Тем не менее, сегодня АЭС Великобритании вырабатывают около 1/6 всей электроэнергии, производимой в стране (16 реакторов суммарной мощностью 10,1 ГВт).
Учитывая, что к 2023 году все реакторы, находящиеся сейчас в эксплуатации, за исключением одного, реактора PWR Сизвелл Б мощностью 1 188 МВт, будут остановлены [1], в планах, сформировавшихся к 2013 году, намечается строительство новых АЭС до 2030 года суммарной мощностью 10–20 ГВт. 
\end{frame}


\begin{frame}{Атомные электростанции и другие источники первичных отходов в Великобритании}
В Великобритании действует 19 энергетических реакторов на 9 площадках. Два перерабатывающих завода и один завод по производству MOX-топлива работают в Селлафильде. Один завод по обогащению урана работает в Капенхёрсте и один завод по производству тепловыделяющих элементов в Спрингфилде. Отходы также образуются в ходе экспериментов JET fusion. 9 энергетических реакторов, 5 исследовательских реакторов, 2 перерабатывающих завода и урановый завод находятся в процессе выведения из экплуатации.


\end{frame}
\end{document}

